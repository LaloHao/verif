% Created 2017-02-01 mié 13:19
% Intended LaTeX compiler: pdflatex
\documentclass[11pt]{/home/hao/dev/org/latex-plantilla/IEEEtran}
\usepackage[spanish, mexico]{babel}
\usepackage{url}
\usepackage{minted}
\addto\captionsspanish{\renewcommand{\contentsname}{Contenido}}
\usepackage[utf8]{inputenc}
\usepackage[T1]{fontenc}
\usepackage{graphicx}
\usepackage{grffile}
\usepackage{longtable}
\usepackage{wrapfig}
\usepackage{rotating}
\usepackage[normalem]{ulem}
\usepackage{amsmath}
\usepackage{textcomp}
\usepackage{amssymb}
\usepackage{capt-of}
\usepackage{hyperref}
\author{Eduardo Vázquez Díaz \\ lalohao@gmail.com}
\date{\today}
\title{Verificación de circuitos digitales con software libre}
\hypersetup{
 pdfauthor={Eduardo Vázquez Díaz \\ lalohao@gmail.com},
 pdftitle={Verificación de circuitos digitales con software libre},
 pdfkeywords={},
 pdfsubject={},
 pdfcreator={Emacs 25.1.1 (Org mode 9.0.3)},
 pdflang={Spanish}}
\begin{document}

\maketitle
\tableofcontents

\begin{abstract}
Se creó una maquina virtual con \emph{Ubuntu Desktop 16.04.1} \uline{LTS} en un
contenedor utilizando el software de virtualizacion \texttt{Qemu}, donde
posteriormente se instaló \texttt{verilator} y \texttt{gtkwave}; a partir de este
sistema se exponen algunas técnicas para simular circuitos con
verilog/C++, además de visualizar las ondas generadas de manera
gráfica.
\end{abstract}

\section{Introducción}
\label{sec:org7a2e3d2}
La importancia de probar los circuitos antes de ser llevados al silicio
puede representar millones de dolares invertidos, sin contar el
tiempo que se llevó a cabo en el diseño, y el que se necesitará
volver a invertir para arreglarlo.

En 1990 el lenguaje de descripción de hardware mas usado era VHDL, a
pesar de que solo tenia constructores básicos para probar los
circuitos (TestBench). Los diseños empezaban a crecer y nuevo
software comercial se creaba para compensar, algunas empresas
invertían horas de trabajo para crear su propio sistema y no pagar
los miles de dolares en licencias, una de ellas llevó a la creación
de Accelera que fue la base de SystemVerilog \cite{spear08_system}.

De la misma manera surgió \texttt{Verilator}, un simulador potente de
Verilog HDL que además es software libre, este compila el código y
lo optimiza para ser simulado rápidamente \cite{verilator-intro}.
\subsection{Virtualización}
\label{sec:org812a018}
La maquina virtual permite encapsular nuestro sistema de
verificación en un contenedor que no será afectado (y que no
afectará) la maquina utilizada, esto elimina errores que podrian ser
causados al tener instalado software que utilice configuraciones
globales (PATHS) como ocurre con \texttt{HSPICE} y \texttt{Questa SIM} por
ejemplo.

\begin{center}
\includegraphics[width=7cm]{virtualizacion.jpg}
\captionof{figure}{\label{fig:org300a892}
Las maquinas virtuales pueden o no conectarse entre ellas o hacia la red externa y pueden ser de diferentes arquitecturas y sistemas operativos independientemente del sistema anfitrion.}
\end{center}

Se le dice anfitrión a la maquina donde se encuentran los
contenedores virtuales, en este caso la anfitriona usa \emph{Arch Linux},
pero esto no afecta a los contenedores ya que estan aislados, esto
aplica de igual manera para Windows o Mac.
\subsection{Requisitos}
\label{sec:orgc5df779}
En la maquina virtual (Ubuntu) se instala el software necesario
para simular, al igual que el editor de texto de su preferencia
para modificar los archivos.
\subsubsection{Verilator y GTKwave}
\label{sec:org0d8baab}
Se pueden instalar desde la terminal con el siguiente comando
\cite{verilator-instalacion}:
\begin{minted}[]{bash}
sudo apt-get install git make autoconf \
     g++ flex bison verilator gtkwave
\end{minted}

\bibliographystyle{plain}
\bibliography{bibliografia.bib}
\end{document}